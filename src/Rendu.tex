% Options for packages loaded elsewhere
\PassOptionsToPackage{unicode}{hyperref}
\PassOptionsToPackage{hyphens}{url}
%
\documentclass[
]{article}
\title{Projet Data Camp: Cinema}
\author{}
\date{\vspace{-2.5em}}

\usepackage{amsmath,amssymb}
\usepackage{lmodern}
\usepackage{iftex}
\ifPDFTeX
  \usepackage[T1]{fontenc}
  \usepackage[utf8]{inputenc}
  \usepackage{textcomp} % provide euro and other symbols
\else % if luatex or xetex
  \usepackage{unicode-math}
  \defaultfontfeatures{Scale=MatchLowercase}
  \defaultfontfeatures[\rmfamily]{Ligatures=TeX,Scale=1}
\fi
% Use upquote if available, for straight quotes in verbatim environments
\IfFileExists{upquote.sty}{\usepackage{upquote}}{}
\IfFileExists{microtype.sty}{% use microtype if available
  \usepackage[]{microtype}
  \UseMicrotypeSet[protrusion]{basicmath} % disable protrusion for tt fonts
}{}
\makeatletter
\@ifundefined{KOMAClassName}{% if non-KOMA class
  \IfFileExists{parskip.sty}{%
    \usepackage{parskip}
  }{% else
    \setlength{\parindent}{0pt}
    \setlength{\parskip}{6pt plus 2pt minus 1pt}}
}{% if KOMA class
  \KOMAoptions{parskip=half}}
\makeatother
\usepackage{xcolor}
\IfFileExists{xurl.sty}{\usepackage{xurl}}{} % add URL line breaks if available
\IfFileExists{bookmark.sty}{\usepackage{bookmark}}{\usepackage{hyperref}}
\hypersetup{
  pdftitle={Projet Data Camp: Cinema},
  hidelinks,
  pdfcreator={LaTeX via pandoc}}
\urlstyle{same} % disable monospaced font for URLs
\usepackage[margin=1in]{geometry}
\usepackage{longtable,booktabs,array}
\usepackage{calc} % for calculating minipage widths
% Correct order of tables after \paragraph or \subparagraph
\usepackage{etoolbox}
\makeatletter
\patchcmd\longtable{\par}{\if@noskipsec\mbox{}\fi\par}{}{}
\makeatother
% Allow footnotes in longtable head/foot
\IfFileExists{footnotehyper.sty}{\usepackage{footnotehyper}}{\usepackage{footnote}}
\makesavenoteenv{longtable}
\usepackage{graphicx}
\makeatletter
\def\maxwidth{\ifdim\Gin@nat@width>\linewidth\linewidth\else\Gin@nat@width\fi}
\def\maxheight{\ifdim\Gin@nat@height>\textheight\textheight\else\Gin@nat@height\fi}
\makeatother
% Scale images if necessary, so that they will not overflow the page
% margins by default, and it is still possible to overwrite the defaults
% using explicit options in \includegraphics[width, height, ...]{}
\setkeys{Gin}{width=\maxwidth,height=\maxheight,keepaspectratio}
% Set default figure placement to htbp
\makeatletter
\def\fps@figure{htbp}
\makeatother
\setlength{\emergencystretch}{3em} % prevent overfull lines
\providecommand{\tightlist}{%
  \setlength{\itemsep}{0pt}\setlength{\parskip}{0pt}}
\setcounter{secnumdepth}{-\maxdimen} % remove section numbering
\ifLuaTeX
  \usepackage{selnolig}  % disable illegal ligatures
\fi

\begin{document}
\maketitle

Master 2 Miage 2021-2022

Alexis Sidate - Gabriel Jestin

\hypertarget{population}{%
\subsection{Population}\label{population}}

\includegraphics{Rendu_files/figure-latex/PopulationFrancaise-1.pdf}

Cette carte présente le nombre de cinéma par Région en France en 2002.
Les zones les plus peuplées en France en 2002 étaient notamment situées
dans les grandes villes (Paris, Lyon, Lille, Marseille \ldots).

\hypertarget{nombre-de-cinuxe9mas}{%
\subsection{Nombre de cinémas}\label{nombre-de-cinuxe9mas}}

\includegraphics[width=0.5\linewidth]{Rendu_files/figure-latex/NbEtablissementParRegion-1}
\includegraphics[width=0.5\linewidth]{Rendu_files/figure-latex/NbEtablissementParRegion-2}

La carte ci-dessus à gauche montre le nombre de cinémas par région, on
peut voir qu'il y a plus de cinémas en Ile-De-France et dans la moitié
Sud de le France. La carte de droite montre que ce ne sont pas forcément
ces zones géographiques qui ont le plus de villes équipées de cinémas
(le nord et le nord-ouest notamment), on peut donc deviner qu'il y a un
grand nombre de cinémas situés dans les mêmes villes (notamment les
grandes villes évoquées précédemment).

\includegraphics[width=0.5\linewidth]{Rendu_files/figure-latex/NbEtablissementParRegionParHabitant-1}
\includegraphics[width=0.5\linewidth]{Rendu_files/figure-latex/NbEtablissementParRegionParHabitant-2}

Sur la carte ci-dessus à droite, on remarque que la région Corse possède
le plus de cinéma par rapport à son nombre d'habitant. Ce chiffre peut
s'expliquer par la très faible population de celle-ci par rapport aux
autres régions. La corse est environ \textbf{16 fois moins} peuplée que
les autres régions de France (en omettant l'Île-De-France). Les cinémas
installés là-bas sont probablement plus fréquentés par les toursites
pendant la période estivale

Sur la seconde carte (sans la Corse), on peut remarquer que les régions
du Sud sont celles qui possèdent le plus de cinémas par rapport à leur
population (avec la région Auvergne-Rhône-Alpes en tête). Ce chiffre
s'explique par le fait que les cinémas installés dans ces régions sont
plus nombreux afin de couvrir la grande superficie de celles-ci. Ces
cinémas possèdent alors moins de salles et de sièges qu'en Île-De-France
car la population y est beaucoup moins concentrée.

\hypertarget{les-entruxe9es}{%
\subsection{Les entrées}\label{les-entruxe9es}}

\begin{verbatim}
## Total des entrées en France en 2002 : 184,459,000 .
\end{verbatim}

\includegraphics{Rendu_files/figure-latex/NbEntreeParHabitant-1.pdf}
Ci-dessus, on observe que les franciliens se rendent plus au cinema que
dans les autres régions (plus de 5 fois par habitant en moyenne°.

On peut voir ci-dessous une très forte domination au niveau des entrées
par la région Ile-De-France qui peut facilement s'expliquer par la très
forte population présente à Paris et en banlieue parisienne ou encore
grâce au tourisme très élevé par rapport aux autres villes françaises.

\includegraphics[width=0.5\linewidth]{Rendu_files/figure-latex/Entrees-1}
\includegraphics[width=0.5\linewidth]{Rendu_files/figure-latex/Entrees-2}

Cette différence est encore plus visible lorsque compare le nombre
d'entrées par département en incluant Paris ou non (voir cartes
ci-dessus).

\includegraphics{Rendu_files/figure-latex/Fauteuils-1.pdf}

La carte ci-dessus du nombre de fauteuils est très similaire à la carte
du nombre d'entrées, cependant elle diffère de la carte du nombre de
communes équipées, on peut notamment voir la différence au niveau de la
Bretagne, ceci peut s'expliquer par la densité de population moins
élevée que dans d'autres régions, obligeant peut-être un plus grand
nombre de villes à s'équiper de cinémas de plus petites tailles.

\hypertarget{les-recettes}{%
\subsection{Les recettes}\label{les-recettes}}

\begin{verbatim}
## Total des recettes en France : €1,027,869,000 .
\end{verbatim}

\begin{longtable}[]{@{}ll@{}}
\caption{Les 5 régions les plus rentables}\tabularnewline
\toprule
Region & Recettes \\
\midrule
\endfirsthead
\toprule
Region & Recettes \\
\midrule
\endhead
Île-de-France & €332,885,000 \\
Auvergne-Rhône-Alpes & €124,711,000 \\
Provence-Alpes-Côte d'Azur & €90,640,000 \\
Grand Est & €80,320,000 \\
Occitanie & €74,578,000 \\
\bottomrule
\end{longtable}

Directement lié au nombre d'entrées par habitant, le total des recettes
est donc naturellement le plus élevée dans la région Ile-De-France.
Ci-dessous, le graphique de gauche met en évidence la part très
conséquente des recettes qui sont faites en Île-De-France par rapport au
reste du pays, à droite on observe que les recettes sont
proportionnelles au nombre d'habitant, mais quand sont même beaucoup
plus élevées Île-De-France.

\includegraphics[width=0.5\linewidth]{Rendu_files/figure-latex/RépartitionRecettes-1}
\includegraphics[width=0.5\linewidth]{Rendu_files/figure-latex/RépartitionRecettes-2}

\end{document}
